%%%%%%%%%%%%%%%%%%%%%%%%%%%%%%%%%%%%%%%%%%%%%%%%%%%%%%%%%%%%%%%%%%%%%%%%%%%%%%%%

% IEEEconf.cls file must exist in the same directory as the TeX file you want to compile
\documentclass[letterpaper, 10 pt, conference]{IEEEconf}

\title{\LARGE \bf
COMPUTER HISTORY\\
\large The Art of Programming
}

\author{Group 22\\
\small Jennifer Quay Minnich\\
\small Group Member 2\\
\small Group Member 3\\
}

% Image/graphics support
\usepackage{graphicx}
\graphicspath{ {./images/} }

% Formatting for lists
\usepackage{enumitem}

% Formatting for media
\usepackage{float}
\restylefloat{table}
\restylefloat{figure}

\begin{document}


\maketitle
\thispagestyle{empty}
\pagestyle{empty}


%%%%%%%%%%%%%%%%%%%%%%%%%%%%%%%%%%%%%%%%%%%%%%%%%%%%%%%%%%%%%%%%%%%%%%%%%%%%%%%%
\section*{ABSTRACT}
\textit{
"I just had to program in order to be happy... 
adding a couple lines of code gave me a real high, 
it must be the way poets feel." 
-Donald Knuth
}

%%%%%%%%%%%%%%%%%%%%%%%%%%%%%%%%%%%%%%%%%%%%%%%%%%%%%%%%%%%%%%%%%%%%%%%%%%%%%%%%
\section{INTRODUCTION}

When we think of programming software, what comes to mind?  Data structures, 
algorithms, zeros and ones? As we explore just what "The Art of Programming" is,
we discover just how sophisticated our programming languages have become 
since the days of Assembly Code and binary. We start to see just how creative
programming actually is. Similar to a composer instructing a symphony to create
the perfect sound, a programmer creates complex literature, conversations between 
humans and machines.

Programming is more solution-based and not based on artistic heuristics, so it is 
not generally considered an art form. But looking closer, programming is in fact art,
of course. It is also science, engineering, and design. Software programming is so 
widespread and important in our society that it is all these things at once.

%%%%%%%%%%%%%%%%%%%%%%%%%%%%%%%%%%%%%%%%%%%%%%%%%%%%%%%%%%%%%%%%%%%%%%%%%%%%%%%%
\section{TIME PERIOD}

You should describe the time period in which your topic was
invented or used here. Also include the context for why your
topic was created or for how it is used. Any specific historical
information should be included here.

%%%%%%%%%%%%%%%%%%%%%%%%%%%%%%%%%%%%%%%%%%%%%%%%%%%%%%%%%%%%%%%%%%%%%%%%%%%%%%%%
\section{COMPUTER HARDWARE}

You should list the specification of any hardware your topic
uses here. If you want make a table here, please label the table
and include discussion on what components are included in the
table and why. See Table
\ref{tbl:example} for an example of a table.
The labels/captions of the table should be put at the bottom
of the table.


\begin{table}[h!]
\begin{center}
\begin{tabular}{||c | c | c | c||} 
\hline
  & Col1 & Col2 & Col3 \\ [0.5ex]
\hline\hline
Row1 & (1,1) & (1,2) & (1,3) \\ 
\hline
Row2 & (2,1) & (2,2) & (2,3) \\
\hline
Row3 & (3,1) & (3,2) & (3,3) \\
\hline
\end{tabular}
\caption{Example of a table}
\label{tbl:example}
\end{center}
\end{table}

You might want to put figures in the document. Please
remember to label them. The labels/captions of the figures
should be put at the bottom of the figure. See Figure
\ref{fig:example} for an example of how to use figures.
You will need to place the figure in an \texttt{images/} folder
in your working directory.

\begin{figure}[h!]
\centering
\includegraphics[width=0.5\textwidth]{spiral.png}
\caption{Example of a figure}
\label{fig:example}
\end{figure} 

\section{COMPUTER SOFTWARE}

Describe the software used for your chosen topic if any,
and state any uses of the software that your topic had.
If your topic does not have or use software, describe why it
doesn't use software and how it functions without it.

\section{CONCLUSION}

Conclude your research paper with any reflections on what you
learned about your topic. Was this what you expected to find?
Did you find any facts that surprised you? You may add other
personal reflections about the topic here.

\section*{REFERENCES}

Below are basic formats for different types of references.

\begin{enumerate}[label={[\arabic*]}]
\item Name of Author, ``Title of chapter in the book,''
Title of The Published Book, xth edition. City of
Publisher, Country if not U.S.
\item Name of Author, “Name of paper,” Abbrev.
Title of Periodical, vol. x, no. x, pp. xxx-xxx,
Abbrev. Month, year.
\item Computer History Museum, (2011),
The Art of Writing Software [Video].
Produced by Hillman & Carr, computerhistory.org.
\item Daed Tech, (Feb, 2016),
Is Programming Art? Erik Deitrich, daedtech.com.
\end{enumerate}

\end{document}

