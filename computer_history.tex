%%%%%%%%%%%%%%%%%%%%%%%%%%%%%%%%%%%%%%%%%%%%%%%%%%%%%%%%%%%%%%%%%%%%%%%%%%%%%%%%

% IEEEconf.cls file must exist in the same directory as the TeX file you want to compile
\documentclass[letterpaper, 10 pt, conference]{IEEEconf}
\usepackage{amsmath}

\title{\LARGE \bf
COMPUTER HISTORY\\
\large The Art of Programming
}

\author{Group 22\\
\small Jennifer Quay Minnich\\
\small Johndenmyr Mendoza\\
\small Elan England\\
}

% Image/graphics support
\usepackage{graphicx}
\graphicspath{ {./images/} }

% Formatting for lists
\usepackage{enumitem}

% Formatting for media
\usepackage{float}
\restylefloat{table}
\restylefloat{figure}

\begin{document}


\maketitle
\thispagestyle{empty}
\pagestyle{empty}


%%%%%%%%%%%%%%%%%%%%%%%%%%%%%%%%%%%%%%%%%%%%%%%%%%%%%%%%%%%%%%%%%%%%%%%%%%%%%%%%
\section*{ABSTRACT}
\textit{
"I just had to program in order to be happy... 
adding a couple lines of code gave me a real high, 
it must be the way poets feel." 
-Donald Knuth
}

%%%%%%%%%%%%%%%%%%%%%%%%%%%%%%%%%%%%%%%%%%%%%%%%%%%%%%%%%%%%%%%%%%%%%%%%%%%%%%%%
\section{INTRODUCTION}

When we think of programming software, what comes to mind?  Data structures, 
algorithms, zeros and ones? As we explore just what "The Art of Programming" is,
we discover just how sophisticated our programming languages have become 
since the days of Assembly Code and binary. We start to see just how creative
programming actually is. Similar to a composer instructing a symphony to create
the perfect sound, a programmer creates complex literature, conversations between 
humans and machines.

Programming is more solution-based and not based on artistic heuristics, so it is 
not generally considered an art form. But looking closer, programming is in fact art,
of course. It is also science, engineering, and design. Software programming is so 
widespread and important in our society that it is all these things at once.

%%%%%%%%%%%%%%%%%%%%%%%%%%%%%%%%%%%%%%%%%%%%%%%%%%%%%%%%%%%%%%%%%%%%%%%%%%%%%%%%
\section{TIME PERIOD}

It is uncertain as to when programming first started. One can argue that programming first started during the birth of mathematics, many years before the $20^{th}$ century, and even before the dawn of the first millennia. Though the period when programming first started is debatable, one thing is for certain, the era of programming is not ending any time soon. Computers and technology vastly encompass our society and our dependency on them has grown. The art of programming is an evolving aspect of human society. As our needs grow, so too do our problems, and with them, our thinking. Humans are great thinkers and problem solvers, we solve problems people from the past cannot comprehend. Programming is an aspect of problem solving and as our society evolves, the art of programming evolves with it. As Jamie Zawinski said, "the software is not complete, until its last user is dead."

%%%%%%%%%%%%%%%%%%%%%%%%%%%%%%%%%%%%%%%%%%%%%%%%%%%%%%%%%%%%%%%%%%%%%%%%%%%%%%%%
\section{COMPUTER HARDWARE}


Programming and hardware go hand-in-hand. You won’t get far running software without a computer
and you won’t get far with a computer that’s running no software. 
As a programmer writes, they keep their mind on the hardware thinking about how their software
will interact with it, addressing limitations and possible issues. They won’t start a huge
operation that will cause the program to run out of memory. The programmer keeps some assets
loaded into memory instead of reaccessing them from the disk during a level change in a video
game. For a program to run smoothly, there must be careful deliberation over how to write the
program effectively, for how it interacts with  the hardware. 
Good programming practices arose in part from the symbiotic relationship between the two
connected things

\section{COMPUTER SOFTWARE}

The art of programming depends highly on software; what good is a machine if the tasks it can complete is highly limited? Programming helps machines become fluid, allowing them to complete more tasks and solve more problems. Programming languages are extremely paramount in software development. The languages vary depending on what tasks need to be accomplished. Consider machine language; computers cannot understand words and diction on their own. They only understand mathematical languages of binaries and hexadecimals. However, humans are also slow at comprehending machine language on its own, so the art of programming needed to evolve. Humans developed programming languages that are easier to understand for both us and the machine. Programming languages allow us to talk to machines. To past societies, such a skill would be incredibly alien, and considering with what we can achieve with it, it is almost godlike. 

\section{CONCLUSION}

We went into our research on The Art of Programming knowing where it's 
history would
eventually bring us: To the world of modern developement. Objects, 
classes, libraries, functions, and every other beautiful tool we use
today. We knew about helpful things we have in programming
, and we knew what comparative torture it was to write a program in 
the past, but we had never thought before of the details of how we went
from, essentially, the stone age to the space age in a matter of a few years.
The practices we see in programming today came from the
evolution and continual improvement of programming from year to year, and
decade to decade. The practices of the 80s were improvements of the 70s, as so on and so on. 
Programmers alive and dead shared their knowledge with the new learning generation, through
their documentation, through their books, or through the standards they help create. 
In order to cause the rapid, absurd grow of computational advancement that we saw in the 20th and
21st century, there was a Mecca of computing knowledge on the bulletin
boards of campus halls and in the notebooks of minds young and old.
These practices grew, became more refined, until they grew so much that what was once considered
refined is now considered the equivalent of rubbing two sticks together.
In this project we learned the history of The Art of Programming.

\section*{REFERENCES}


\begin{enumerate}[label={[\arabic*]}]
\item Knowles, A, (31 Aug 2020), \emph{CSE 101 Lecture}, Lecture
\item Computer History Museum, (2011),
The Art of Writing Software [Video].
Produced by Hillman and Carr, computerhistory.org.
\item Daed Tech, (Feb, 2016),
Is Programming Art? Erik Deitrich, daedtech.com.
\item Wikipedia, (Oct, 2020),
Programming, ``Wikipedia.com,''
\end{enumerate}

\end{document}

